\documentclass[a4paper,oneside,12pt]{book}
\usepackage[brazil]{babel}
\usepackage{srcltx}
\usepackage{graphicx}
\usepackage[T1]{fontenc}
\usepackage{amsthm,amsfonts}
%\usepackage{fancybox}
\setlength{\textwidth}{6 in}
\setlength{\textheight}{23 cm}
\evensidemargin 0 in
\oddsidemargin 0.25 in
\sloppy

\newcommand{\binv}{\textbackslash}
\newcommand{\pot}{\textasciicircum}
\newcommand{\til}{\textasciitilde}
\newcommand{\field}[1]{\mathbb{#1}}

\newcommand{\N}{\field{N}}
\newcommand{\R}{\field{R}}
\newcommand{\V}{\field{V}}
\newcommand{\W}{\field{W}}
\newcommand{\C}{\field{C}}
\newcommand{\sen}{\mathrm{sen}}
\newcommand{\dist}{\mathrm{dist}}

\def\href#1#2{\special{html:<a href="#1">}{#2}\special{html:</a>}}
  % Macro for http reference inclusion, per hypertex.
\def\tthdump{} % Do nothing.

\tthdump\def\latexhtml#1#2{#1}
%%tth:\def\latexhtml#1#2{\special{html:#2}}
\tthdump\def\eqc#1{$$#1$$}
%%tth:\def\eqc#1{\begin{center}\setlength{\unitlength}{1pt}\begin{picture}(50,12)\put(0,0){$\displaystyle #1$}\end{picture}\end{center}}
\tthdump\def\eq#1{$#1$}
%%tth:\def\eq#1{\setlength{\unitlength}{1pt}\begin{picture}(50,12)\put(0,0){$#1$}\end{picture}}
\def\eqn#1{\begin{equation}#1\end{equation}}
%%tth:\def\eqn#1{\begin{equation}\begin{center}\eq{\displaystyle #1}\end{center}\end{equation}}
\tthdump\newcommand{\hyperref}[2]{#1 \ref{#2}}
%%tth:\def\hyperref#1#2{<A href="\##2">#1</A>}
\tthdump\newcommand{\hyperpageref}[2]{#1 \pageref{#2}}
%%tth:\def\hyperpageref#1#2{\special{html:<A href="\##2">#1</A>}}

\renewcommand{\rmdefault}{cmss}
\bibliographystyle{plain}

\includeonly{matfin}

\begin{document}
%\fontfamily{cmss}\selectfont
\title{T�picos em Matem�tica:\\
%        Matem�tica Financeira e Aplica��es de �lgebra Linear
        Computa��o no Ensino de Matem�tica e Matem�tica Financeira
        }
\author{Reginaldo J. Santos\\
        Departamento de Matem�tica\\
        Instituto de Ci�ncias Exatas\\
        Universidade Federal de Minas Gerais\\
        \href{http://www.mat.ufmg.br/~regi}
        {\texttt{http://www.mat.ufmg.br/\~{}regi}}\\
        \\[1.5in]
        }
\date{1o. Semestre de 1998}

\newtheorem{teo}{Teorema}[section]
\newtheorem{lema}[teo]{Lema}
\newtheorem{cor}[teo]{Corol�rio}
\newtheorem{prop}[teo]{Proposi��o}
\frontmatter
\maketitle

\renewcommand{\contentsname}{Conte�do}
\tableofcontents

\mainmatter
%\chapter{Introdu��o ao Windows}
\INCLUDE{intwin}

%\chapter{Introdu��o ao MATLAB}
\INCLUDE{intmatl}

%\chapter{Introdu��o ao \LaTeX}
\INCLUDE{intlat}

%\chapter{Matem�tica Financeira}
\INCLUDE{matfin}


\end{document}

% Este é um pequeno arquivo fonte para o LaTeX
% Use este arquivo como modelo para fazer seus próprios arquivos  LaTeX.
% Tudo que está à direita de um  %  é um comentário e é ignorado pelo LaTeX.
%
\documentclass[a4paper,12pt]{article}% Seu arquivo fonte precisa conter
\usepackage[brazil]{babel}           % estas quatro linhas
\usepackage[utf8]{inputenc}          % além do comando \end{document}
\begin{document}                     % no fim.

\section{Texto, Comandos e Ambientes} % Este comando faz o título da seção.

Um arquivo fonte do \LaTeX\ contém além do texto a ser processado,
comandos que indicam como o texto deve ser processado. Palavras
são separadas por um ou mais espaços. Parágrafos são separados por
uma ou mais linhas em branco. A saída não é afetada por espaços
extras ou por linhas em branco extras. A maioria dos comandos do
\LaTeX \ são iniciados com o caracter $\backslash$. Uma
$\backslash$ sozinha produz um espaço. Um ambiente é uma região do
texto
que tem um tratamento especial. Um ambiente é iniciado com\\
\texttt{$\backslash$begin\{nome do ambiente\}} e terminado por
\texttt{$\backslash$end\{nome do ambiente\}}.

%Aspas são digitadas assim:
``Texto entre aspas''.

%Texto em itálico deve ser digitado como:
\textit{Isto está em itálico}.

%Texto em negrito deve ser digitado como:
\textbf{Isto está em negrito}.

\subsection{Um aviso}  % Este comando faz o título da subseção.

Lembre-se de não digitar nenhum dos 10 caracteres especiais
%                &   $   #   %   _   {   }   ^   ~   \
\& \$ \# \% \_ \{ \} \^{} \ \~{}\ $\backslash$ exceto como um
comando!
\end{document}   % O arquivo fonte termina com este comando.

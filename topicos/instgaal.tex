%%tth:\def\latexhtml#1#2{\special{html:#2}}
%%tth:\def\eqc#1{\begin{center}\setlength{\unitlength}{1pt}\begin{picture}(50,12)\put(0,0){$\displaystyle #1$}\end{picture}\end{center}}
%%tth:\def\eq#1{\setlength{\unitlength}{1pt}\begin{picture}(50,12)\put(0,0){$#1$}\end{picture}}
%%tth:\def\hyperref#1#2{\special{html:<A href="\##2">#1</A>}}
%%tth:\def\hyperpageref#1#2{\special{html:<A href="\##2">#1</A>}}

%%tth:\title{Instala��o do Pacote GAAL
%%tth:        }
%%tth:\author{Reginaldo J. Santos\\
%%tth:        Departamento de Matem�tica\\
%%tth:        Instituto de Ci�ncias Exatas\\
%%tth:        Universidade Federal de Minas Gerais\\
%%tth:        \href{http://www.mat.ufmg.br/~regi}
%%tth:        {\texttt{http://www.mat.ufmg.br/\~{}regi}}
%%tth:        }
%%tth:\date{11 de novembro de 1998}


\begin{document}

\begin{enumerate}
\item
Entre em \textbf{Meu Computador} (clicando duas vezes com o
bot�o esquerdo do mouse sobre \textbf{Meu Computador} no
desktop). Entre em \textbf{drive C:}, \textbf{MATLAB} e
depois em \textbf{TOOLBOX}. Crie uma pasta chamada
\texttt{gaal}.

\item \tthdump{Inicialize o \textbf{Netscape}.\\ Escreva na janela
\textbf{Go to (Ir para)} o endere�o
%\href{http://www.mat.ufmg.br/~regi}
\texttt{http://www.mat.ufmg.br/\~{}regi} e
tecle \textbf{Enter}.
$$\includegraphics[width=4.3in,height=1in] {netscape.gif}$$}
Siga as instru��es que est�o nesta p�gina
%%tth:\href{http://www.mat.ufmg.br/~regi}{\texttt{http://www.mat.ufmg.br/\~{}regi}}~
para fazer o ``download'' do pacote \texttt{gaal}, ou seja,
para trazer o pacote para o seu computador.


Salve na pasta rec�m-criada \texttt{gaal}.

\item
Use \textbf{Alt+Tab} para alternar para a janela com o
conte�do de
\texttt{C:$\backslash$MATLAB$\backslash$TOOLBOX},
 clique duas vezes com o bot�o
esquerdo do mouse sobre o �cone da pasta \texttt{gaal} e
depois fa�a o mesmo sobre o �cone do pacote que voc� fez
``download'', \texttt{gaal.exe}. O pacote ser�
descompactado.

\item
Inicialize o MATLAB, se j� n�o o tiver feito.

\item
Na barra de ferramentas
$$\includegraphics[width=4.37in,height=0.3in]{matlabbt.gif}$$
clique com o bot�o esquerdo do mouse no
bot�o
\includegraphics[width=0.3in,height=0.3in]{matlabpb.gif}

\item
Clique no bot�o \textbf{Add to Path...}.
$$\includegraphics[width=3.48in,height=2in]{matlabp.gif}$$
Clique duas vezes  com o bot�o esquerdo do mouse em
\texttt{toolbox} e depois o mesmo em
\texttt{gaal}. Depois clique em \textbf{Add to Back}
$$\includegraphics[width=2.56in,height=2in]{matlabap.gif}$$
\item
Clique em \textbf{OK}. Depois, em \textbf{Save Settings} e
por �ltimo em \textbf{Close}


\item
Verifique se o MATLAB adicionou o pacote \texttt{gaal} aos
outros, digitando no prompt \texttt{winhelp}. Ele deve
aparecer na �ltima linha ou na primeira. Caso contr�rio
repita o processo acima, com mais cuidado.

\item Para informa��es sobre o pacote \texttt{gaal}
digite no prompt do MATLAB \\
\texttt{help gaal}

\end{enumerate}


\end{document}
